\documentclass{article}
\usepackage{cite}


\begin{document}
\author{Max Horowitz-Gelb}
\title{Active Learning For DBSI: DNA Binding Site Identifier }
\maketitle

\section*{Abstract}
DBSI is a structure based method for predicting the positions of protein interaction sites associated with DNA binding \cite{dbsi}. Here I present a method for applying active learning to the training of DBSI. This method optimizes the training of DBSI in a way that considers the batch style form of labelled data collection necessary when creating a training set. This method shows slight improvements in efficiency in comparison to naive methods. 
\section*{Introduction}
For area under an ROC curve, DBSI has been shown to achieve $~88\%$, a high degree of separability. The score was achieved by training DBSI on a set of 263 unique proteins. The model as a result of this training is now accessible to anyone on a public server \cite{dbsi_server}. The quality of this model could be improved further by training with more labelled data.  
\section*{Methods}
\cite{active_learning}
\section*{Results}
\section*{Conclusion}

\bibliography{mybib}{}
\bibliographystyle{plain}
\end{document}
